
\documentclass[preprint,10pt]{elsarticle}

\usepackage{epsfig}
\usepackage{amsmath,amssymb,bm}
\usepackage{hyperref}


\journal{...}

\begin{document}

\begin{frontmatter}
\title{Probing photonic content of the proton using photon-induced dilepton production in $p+Pb$ collisions at the LHC}


\author{M. Dyndal}
\address{DESY}
\author{A. Glazov}
\address{DESY}
\author{M. Luszczak}
\address{...}
\author{R. Sadykov}
\address{...}



\begin{abstract}
XX
\end{abstract}


%%\keywords{QED, Equivalent Photon Approximation, LHC}

\end{frontmatter}


%%%%%%%%%%%%%%%%%%%%%%%%%%%%%%%%%%
\section{Introduction}
%%%%%%%%%%%%%%%%%%%%%%%%%%%%%%%%%%

A significant fraction of proton-proton collisions at the LHC involves quasi-real photon interactions,
where the photons are emitted by both protons. The proton-proton collision is then transformed into a photon-photon interaction
and the protons are deflected at small angles. At LHC energies, these reactions
 occur at
energies well beyond the electroweak energy scale. They offer an interesting field of research linked to photon-photon interactions,
 where the available effective luminosity is small, relative to parton-parton interactions, but is compensated
by better known initial conditions and usually simpler final states. 

...~\cite{Chatrchyan:2011ci}


\section{Results}

... add your contributions in separate .tex files


\section*{References}
%\clearpage
\bibliographystyle{elsarticle-harv}
\bibliography{main}


\end{document}

