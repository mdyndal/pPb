%%%%%%%%%%%%%%%%%%%%%%%%%%%%%%%%%%
\section{Example experimental configuration and possible background sources}
%%%%%%%%%%%%%%%%%%%%%%%%%%%%%%%%%%
\label{sec:experiment}

We assume collision setup from recent $p+\textrm{Pb}$ run at the LHC, carried out at the centre-of-mass energy per nucleon pair $\sqrt{s_{N N}} = 8.16$~\TeV.
Since the energy per nucleon in the proton beam is larger than in the lead beam, the nucleon--nucleon centre-of-mass system has a rapidity in the laboratory frame of $y = 0.465$.

As an example of method's applicability, we will use the geometry of ATLAS~\cite{Aad:2008zzm} and CMS~\cite{Chatrchyan:2008aa} detectors in the following.
We consider only the dimuon channel, however the integrated results for $ee$ and $\mu\mu$ channels can be obtained by simply multiplying the dimuon cross-sections by a factor of two.


We start by applying a minimum transverse momentum requirement of 4~\GeV\ to both muons.
This requirement is imposed to ensure high lepton reconstruction and triggering efficiency.
Moreover, due to limited acceptance of the detectors, each muon is required to have a pseudorapidity ($\eta^{\ell}$) that satisfies $|\eta^{\ell}|<2.4$.
Our calculations are carried out for a minimum dilepton invariant mass of $m_{\ell^+\ell^-} = 10$~\GeV. 
Such a choice is due to removal of possible contamination from $\Upsilon(\rightarrow \ell^+\ell^-)$ photoproduction.
A summary of all selection requirements is presented in Table~\ref{tab:fidRegion}.

\begin{table}[t!]
  \begin{center}
    \begin{tabular}{|l|c|}
      \hline 
    Variable  & Requirement \\ \hline
    lepton transverse momentum, $p_{\textrm{T}}^{\ell}$ & $>4$~\GeV \\
    lepton pseudorapidity, $|\eta^\ell|$ & $<2.4$ \\
    dilepton invariant mass, $m_{\ell^+\ell^-}$ & $>10$~\GeV  \\
      \hline 
    \end{tabular}
  \end{center}
  \caption{Definition of the fiducial region used in the studies.}
  \label{tab:fidRegion}
\end{table}

Possible background for this process can arise from inclusive lepton-pair production, e.g. from Drell--Yan process~\cite{Drell:1970wh,Aad:2015gta,Khachatryan:2015pzs,Alice:2016wka}.
This processes would lead to disintegration of the incoming ion, and zero-degree calorimeters (ZDC)~\cite{Dellacasa:1999ke,ATLAS:2007aa} can be used to veto very-forward-going neutral fragments which would allow this background 
to be reduced fully.
Another background can arise from diffractive interactions, hence possibly mimicking the signal topology.
However, since the Pb nucleus is a fragile object (with the nucleon binding energy of just 8 MeV) even the softest diffractive interaction will likely result in the emission of a few nucleons from the ion, detectable in the ZDC.

Another background category is the photon-induced process with a resolved photon, i.e. 
$\gamma p\rightarrow Z/\gamma^*+X$ reaction.
Here, the rapidity gap is expected to be smaller than in the signal process due to the additional particle production associated with the ``photon remnant''.
Any other residual contamination of this process can be controlled using a dedicated sample, with a dilepton invariant mass around the $Z$-boson mass.



