%%%%%%%%%%%%%%%%%%%%%%%%%%%%%%%%%%
\section{Results with collinear photon-PDFs}
%%%%%%%%%%%%%%%%%%%%%%%%%%%%%%%%%%

We start with the calculation of the elastic contribution, $p+\textrm{Pb}\rightarrow p+\textrm{Pb} \ell\ell$.
In this case the photon flux becomes:
\begin{equation}
f_\gamma^{p}(x,\mu) = f_\gamma^{p}(x) 
\end{equation}
and the following parameterization is used~\cite{}:
\begin{equation}
f_\gamma^{p}(x) = \frac{\alpha}{\pi}
\left(
\frac{1-x+0.5x^2}{x}
\right)
\left(
\frac{A+3}{A-1}\log{A}-\frac{17}{6}-\frac{4}{3A}+\frac{1}{6A^2}
\right)~,
\end{equation}
where $A = 1+\frac{Q_0^2(1-x)}{x m_p^2}$ and $Q_0^2 = 0.71$~\GeV$^2$.

The results for the elastic case are cross-checked with the calculation from STARlight MC and a good agreement is found:
$\sigma_{fid}^{\textrm{el}} = 17.5$~nb, whereas $\sigma_{fid}^{\textrm{STARlight}} = 17.0$~nb.
Both calculations are also corrected by a factor $S^2=0.96$ which takes into account the requirement that there be no hadronic interactions between the proton and the ion. This is calculated using STARlight, where the hard-sphere proton--nucleus requirement~\cite{Klein:2016yzr} is used.

Next, for the inelastic case ($\gamma p\rightarrow \ell\ell X$), several recent parameterizations of the photon parton distributions are studied: CT14qed~\cite{}, LUXqed17~\cite{} and NNPDF3.1luxqed~\cite{}.
Comparison of several lepton kinematic distributions between different photon-PDFs are shown in Fig.~\ref{}.
The integrated fiducial cross-sections are summarized in Tab.~\ref{}.

(some discussion here...)

It should be made clear, that the calculations with collinear photons (at lowest order) produce leptons that are back-to-back in transverse kinematics. Therefore, to take the effect of inelastic photon virtuality into account, a dedicated parton shower algorithm should be used.

(mention we don't want to do this; we would rather stick to kt factorization)
