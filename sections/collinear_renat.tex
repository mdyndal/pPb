%%%%%%%%%%%%%%%%%%%%%%%%%%%%%%%%%%
\section{Results with collinear photon-PDFs}
%%%%%%%%%%%%%%%%%%%%%%%%%%%%%%%%%%

We start with the calculation of the elastic contribution, $p+\textrm{Pb}\rightarrow p+\textrm{Pb}+ \ell^+\ell^-$.
In this case the photon flux becomes:
\begin{equation}
\gamma^{p}_{el}(x, Q^2) = \gamma^{p}_{el}(x) 
\end{equation}
and the following parameterization is used~\cite{Budnev:1974de}:
\begin{equation}
\gamma^{p}_{el}(x)  = \frac{\alpha_{\rm em}}{\pi}
\left(
\frac{1-x+0.5x^2}{x}
\right)
\left(
\frac{A+3}{A-1}\log{A}-\frac{17}{6}-\frac{4}{3A}+\frac{1}{6A^2}
\right)~,
\end{equation}
where $A = 1+\frac{Q_0^2(1-x)}{x m_p^2}$ and $Q_0^2 = 0.71$~\GeV$^2$. This parameterization is a good analytical approximation of Eq.~\ref{proton_el_flux} integrated over $Q^2$.

The results for the elastic case are cross-checked with the calculation from STARlight MC and a good agreement between the fiducial cross-sections is found:
$\sigma_{fid}^{\textrm{el}} = 17.5$~nb, whereas $\sigma_{fid}^{\textrm{STARlight}} = 17.0$~nb.
Both calculations are also corrected by a factor $S^2=0.96$ which 
%takes into account the requirement that there be no hadronic interactions between the proton and the ion. This 
is calculated using STARlight, where the hard-sphere proton--nucleus requirement~\cite{Klein:2016yzr} is used.

Next, for the inelastic case ($\gamma p\rightarrow \ell^+\ell^- + X$), several recent parameterizations of the photon parton distributions are studied: CT14qed~\cite{Schmidt:2015zda}, HKR16qed~\cite{Harland-Lang:2016kog}, LUXqed17~\cite{Manohar:2017eqh} and NNPDF3.1luxQED~\cite{Bertone:2017bme}. 
One should note that all of these PDF sets include both elastic and inelastic parts of the photon spectrum.
All predictions are scaled by $S^2=0.95$, again derived from STARlight.
Comparison of several lepton kinematic distributions between different photon-PDFs is shown in Fig.~\ref{fig:inc_cut}.

%\begin{figure}[h!]
%\includegraphics[width=0.4\textwidth]{figures/Mll_elastic.pdf}
%\includegraphics[width=0.4\textwidth]{figures/RatioMll_elastic.pdf}
%\includegraphics[width=0.4\textwidth]{figures/Yll_elastic.pdf}
%\includegraphics[width=0.4\textwidth]{figures/RatioYll_elastic.pdf}
%\includegraphics[width=0.4\textwidth]{figures/pTl_elastic.pdf}
%\includegraphics[width=0.4\textwidth]{figures/RatiopTl_elastic.pdf}
%\includegraphics[width=0.4\textwidth]{figures/etal_elastic.pdf}
%\includegraphics[width=0.4\textwidth]{figures/Ratioetal_elastic.pdf}
%\caption{Elastic distributions (the only cut is on leptons $p_T$)}
%\label{fig:elastic}
%\end{figure}
%
%\begin{figure}[h!]
%\includegraphics[width=0.4\textwidth]{figures/Mll_elastic_cut.pdf}
%\includegraphics[width=0.4\textwidth]{figures/RatioMll_elastic_cut.pdf}
%\includegraphics[width=0.4\textwidth]{figures/Yll_elastic_cut.pdf}
%\includegraphics[width=0.4\textwidth]{figures/RatioYll_elastic_cut.pdf}
%\includegraphics[width=0.4\textwidth]{figures/pTl_elastic_cut.pdf}
%\includegraphics[width=0.4\textwidth]{figures/RatiopTl_elastic_cut.pdf}
%\includegraphics[width=0.4\textwidth]{figures/etal_elastic_cut.pdf}
%\includegraphics[width=0.4\textwidth]{figures/Ratioetal_elastic_cut.pdf}
%\caption{Elastic distributions (fiducial region)}
%\label{fig:elastic_cut}
%\end{figure}
%
%\begin{figure}[h!]
%\includegraphics[width=0.4\textwidth]{figures/Mll_inc.pdf}
%\includegraphics[width=0.4\textwidth]{figures/RatioMll_inc.pdf}
%\includegraphics[width=0.4\textwidth]{figures/Yll_inc.pdf}
%\includegraphics[width=0.4\textwidth]{figures/RatioYll_inc.pdf}
%\includegraphics[width=0.4\textwidth]{figures/pTl_inc.pdf}
%\includegraphics[width=0.4\textwidth]{figures/RatiopTl_inc.pdf}
%\includegraphics[width=0.4\textwidth]{figures/etal_inc.pdf}
%\includegraphics[width=0.4\textwidth]{figures/Ratioetal_inc.pdf}
%\caption{Inclusive distributions (the only cut is on leptons $p_T$)}
%\label{fig:inc}
%\end{figure}

\begin{figure}[h!]
\includegraphics[width=0.43\textwidth]{figures/Mll_inc_cut.pdf}
\includegraphics[width=0.43\textwidth]{figures/RatioMll_inc_cut.pdf}
\includegraphics[width=0.43\textwidth]{figures/Yll_inc_cut.pdf}
\includegraphics[width=0.43\textwidth]{figures/RatioYll_inc_cut.pdf}
\includegraphics[width=0.43\textwidth]{figures/pTl_inc_cut.pdf}
\includegraphics[width=0.43\textwidth]{figures/RatiopTl_inc_cut.pdf}
\includegraphics[width=0.43\textwidth]{figures/etal_inc_cut.pdf}
\includegraphics[width=0.43\textwidth]{figures/Ratioetal_inc_cut.pdf}
\caption{Differential cross sections in the fiducial region for $p+\textrm{Pb}\rightarrow \textrm{Pb} + \ell^+\ell^- + X$ production at $\sqrt{s_{N N}} = 8.16$~\TeV\ for different collinear photon PDF sets.
Four differential distributions are shown (from top to bottom): invariant mass of lepton pair, pair rapidity, transverse momentum of negatively-charged lepton and its pseudorapidity. Figures on the right show the ratios to LUXqed17 PDF.}
\label{fig:inc_cut}
\end{figure}

The integrated fiducial cross-sections are summarized in Tab.~\ref{fig:xs}.

(some discussion here...)

\begin{table}[b]
\begin{center}
\begin{tabular}{|l|l|l|}
\hline
Contribution & $p_T^{\ell} > 4$ \GeV & $p_T^{\ell}  > 4$ \GeV, $|\eta^{\ell}| < 2.4$,\\
& & $m_{\ell^+\ell^-} > 10$ \GeV\\
\hline
%$\gamma^{p}_{\rm{el}}$ [$b_{min}=0.7fm$] & 45.5(2) nb & 17.3(1) nb\\
%\hline
$\gamma^{p}_{\rm{el}}$  & 44.9(1) nb & 17.5(1) nb\\ % [Electric]
%\hline
%$\gamma^{p}_{\rm{el}}$ [DZ] & 53.3(1) nb & 19.4(1) nb\\
%\hline
%$\gamma^{p}_{\rm{el}}$[CT14qed\_proton] & 48.4(1) nb & 17.6(1) nb\\
\hline
$\gamma^{p}_{\rm{el}} + \gamma^{p}_{\rm{inel}}$ [CT14qed\_inc] & 98.0(1) nb & 39.7(1) nb\\
\hline
$\gamma^{p}_{\rm{el}} + \gamma^{p}_{\rm{inel}}$ [LUXqed17] & 105.8(1) nb & 44.1(1) nb\\
\hline
$\gamma^{p}_{\rm{el}} + \gamma^{p}_{\rm{inel}}$ [NNPDF3.1luxQED] & 115.6(1) nb & 45.9(1) nb\\
\hline
$\gamma^{p}_{\rm{el}} + \gamma^{p}_{\rm{inel}}$ [HKR16qed] & 121.6(1) nb & 49.4(1) nb\\
\hline
\end{tabular}
\end{center}
\caption{Integrated fiducial cross sections for $p+\textrm{Pb}\rightarrow \textrm{Pb} + \ell^+\ell^- + X$ production at $\sqrt{s_{N N}} = 8.16$~\TeV\ for different collinear photon PDF sets. 
An effect of applying only $p_T^{\ell}$ requirement is shown in second column.
For comparison, the cross section for purely elastic contribution is also shown.}
\label{fig:xs}
\end{table}

It should be made clear, that the calculations with collinear photons (at lowest order) produce leptons that are back-to-back in transverse kinematics. Therefore, to take the effect of inelastic photon virtuality into account, a dedicated parton shower algorithm should be used.

(mention we don't want to do extra PS; we would rather stick to kt factorization)
