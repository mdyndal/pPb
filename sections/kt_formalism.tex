%--------------------------------------------------
\subsection{$k_T$-factorization approach}
%--------------------------------------------------

%In this approach we start from the Feynman diagrams shown in Fig.\ref{fig:mateusz},
%and exploit the high-energy kinematics.
%Let the four-momenta of incoming protons be denoted $p_A,p_B$. At high energies 
%the proton masses can be neglected, so that $p_A^2 = p_B^2 =0, \,  2 (p_A\cdot p_B) =s$.
%
%
%The photon-fusion production mechanism in leptonic and hadronic reactions
%is in great detail reviewed in \cite{Budnev:1974de}, where also many original
%references can be found. In the most general form, the invariant cross section
%is written as a convolution of density matrices of photons in the beam particles,
%and helicity amplitudes for the $\gamma^* \gamma^* \to l^+ l^-$ process.
%In a high energy limit, where dileptons carry only a small fraction of the 
%total center-of-mass energy, the density-matrix structure can be very much
%simplified, and there emerges a $k_T$-factorization representation of the
%cross section \cite{daSilveira:2014jla}.
%
%The unintegrated photon fluxes introduced in \cite{daSilveira:2014jla}
%can be expressed in terms of the hadronic tensor as 
%%%
%\begin{eqnarray}
% {\cal{F}}^{{\rm{in.el}}}_{\gamma^* \leftarrow A} (z,\bq) = {\alpha_{\rm{em}}\over \pi}  \, (1-z) \, 
%\Big( {\bq^2 \over \bq^2 + z (M_X^2 - m_A^2) + z^2 m_A^2  }\Big)^2  \, 
%\cdot {p_B^\mu p_B^\nu \over s^2} \, W^{\rm{in,el}}_{\mu \nu}(M_X^2,Q^2) dM_X^2 \, . \nonumber \\
%\end{eqnarray}
%%%
%These unintegrated fluxes enter the cross section for dilepton production as
%%%
%\begin{eqnarray}
% {d \sigma^{(i,j)} \over dy_1 dy_2 d^2\bp_1 d^2\bp_2} &&=  \int  {d^2 \bq_1 \over \pi \bq_1^2} {d^2 \bq_2 \over \pi \bq_2^2}  
% {\cal{F}}^{(i)}_{\gamma^*/A}(x_1,\bq_1) \, {\cal{F}}^{(j)}_{\gamma^*/B}(x_2,\bq_2) 
%{d \sigma^*(p_1,p_2;\bq_1,\bq_2) \over dy_1 dy_2 d^2\bp_1 d^2\bp_2} \, , \nonumber \\ 
%\label{eq:kt-fact}
%\end{eqnarray}
%%%
%where the indices $i,j \in \{\rm{el}, \rm{in} \}$ denote elastic or inelastic final states.
%The longitudinal momentum fractions of photons are obtained from the rapidities 
%and transverse momenta of final state leptons as:
%%%
%\begin{eqnarray}
%x_1 &=& \sqrt{ {\bp_1^2 + m_l^2 \over s}} e^{y_1} +  \sqrt{ {\bp_2^2 +
%    m_l^2 \over s}} e^{y_2} 
%\; , \nonumber \\
%x_2 &=& \sqrt{ {\bp_1^2 + m_l^2 \over s}} e^{-y_1} +  \sqrt{ {\bp_2^2 + m_l^2 \over s}} e^{-y_2} \, .
%\end{eqnarray}
%%%
%The explicit form of the off-shell cross section $d \sigma^*(p_1,p_2;\bq_1,\bq_2)/ dy_1 dy_2 d^2\bp_1 d^2\bp_2$ can be found in
%Refs. \cite{daSilveira:2014jla,Luszczak:2015aoa}. 
%

At lowest order, the calculations with collinear photons  produce leptons that are back-to-back in transverse kinematics. 
Therefore, to take the effect of transverse momentum smearing into account, a dedicated parton shower algorithms are usually used.

In the $k_T$ factorization approach, one can parametrize the $\gamma^*p \rightarrow X$ vertices in terms of the proton structure functions. The photons from inelastic production have transverse momenta and non-zero virtualities $Q^2$ and the unintegrated photon distributions are used, in contrast to collinear distributions.
In the DIS limit, the unintegrated inelastic photon flux can be obtained using the following equation~\cite{daSilveira:2014jla, Luszczak:2015aoa}:

%\begin{eqnarray}
%{d \gamma^p_{inel}(x,Q^2) \over d \log Q^2} =&& {\alpha_{\rm{em}} \over 2 \pi} \int_x^1 {dy \over y} 
%P_{\gamma \leftarrow q}({x\over y}) 
%{F_2(y, Q^2) \over y} \left(1- {x\over y}\right)~,
%\end{eqnarray}
%where $F_2(x, Q^2)$ is the standard proton structure function, and the splitting function $P_{\gamma \leftarrow q}(x)$ is given as
%\begin{eqnarray}
%P_{\gamma \leftarrow q}(x) = {1+(1-x)^2 \over x}
%\label{kt_factorization_first}
%\end{eqnarray}
%%%
\begin{eqnarray}
\gamma^p_{inel}(x,\vec{q}_T) = {1\over x} 
{1 \over \pi q_T^2} \, \int_{M^2_{\rm thr}} dM_X^2 {\cal{F}}^{\mathrm{in}}_{\gamma^* \leftarrow p} (x,\vec{q}_T,M^2_X) \, ,
\end{eqnarray}
%%
and we use the functions $ {\cal{F}}^{\mathrm{in}}$ from \cite{Budnev:1974de, Luszczak:2018ntp}:
%%%
\begin{eqnarray}
{\cal{F}}^{\mathrm{in}}_{\gamma^* \leftarrow p} (x,\vec{q}_T) &=& {\alpha_{\rm em} \over \pi} 
\Big\{(1-x) \Big( {\vec{q}_T^2 \over \vec{q}_T^2 + x (M_X^2 - m_p^2) + x^2 m_p^2  }\Big)^2  
{F_2(x_{\rm Bj},Q^2) \over Q^2 + M_X^2 - m_p^2}  \nonumber \\
&+& {x^2 \over 4 x^2_{\rm Bj}}  
{\vec{q}_T^2 \over \vec{q}_T^2 + x (M_X^2 - m_p^2) + x^2 m_p^2  }
{2 x_{\rm Bj} F_1(x_{\rm Bj},Q^2) \over Q^2 + M_X^2 - m_p^2} \Big\} \, .
%\label{eq:flux_in}
\end{eqnarray}
%%%
The virtuality $Q^2$ of the photon depends on transverse momentum and the remnant mass $M_X$:
%%%
\begin{eqnarray}
Q^2 =  {\vec{q}_T^2 + x (M_X^2 - m_p^2) + x^2 m_p^2 \over (1-x)} \, ,
\end{eqnarray}
%%
and the proton structure functions require the argument
%%%
\begin{eqnarray}
x_{\rm Bj} = {Q^2 \over Q^2+M^2_X -m_p^2}.
\end{eqnarray}
%%
Notice that in (\ref{eq:flux_in}) instead of 
$F_2(x_{\rm Bj},Q^2),F_1(x_{\rm Bj},Q^2)$, 
we use in practice the pair $F_2(x_{\rm Bj},Q^2),F_L(x_{\rm Bj},Q^2)$, where
%%%
\begin{eqnarray}
F_L(x_{\rm Bj},Q^2) = \Big( 1 + {4 x_{\rm Bj}^2 m_p^2 \over Q^2} \Big) F_2(x_{\rm Bj},Q^2) - 2 x_{\rm Bj} F_1(x_{\rm Bj},Q^2)
\end{eqnarray}
%%%
is the longitudinal structure function of the proton.

These unintegrated fluxes enter the $p+\textrm{Pb}\rightarrow \textrm{Pb} + \ell^+\ell^- + X$ production cross section as

\begin{equation}
\sigma = S^2 \int dx_p dx_{\rm Pb} d\vec{q}_T \Big[
\left ( \gamma^{p}_{el}(x_p, \vec{q}_T) + \gamma^{p}_{inel}(x_p,\vec{q}_T) \right)
 \gamma^{\rm Pb}_{el}(x_{\rm Pb})
\sigma_{\gamma^{*}  \gamma \rightarrow \ell^+ \ell^-}(x_p, x_{\rm Pb}, \vec{q_T})  \Big]~,
\label{kt_factorization_formula}
\end{equation}
%
where $\sigma_{\gamma^{*} \gamma \rightarrow \ell^+ \ell^-}$ is the off-shell elementary cross-section~\cite{daSilveira:2014jla} and 
notice that for the elastic case at $x_p \ll 1$ we have $Q^2 \approx \vec{q}_T^2$.
The elastic flux can be obtained from eq.(1) after multiplying by
$(1-x)/(\pi Q^2)$, which takes into account the jacobian from
$dQ^2/Q^2$ to $d\vec{q}$. Please note also, that while fluxes do not depend on the direction of $\vec{q}$, averaging over directions
of $\vec{q}$ in the off-shell cross section replaces the average over photon polarizations in the collinear case.
