%--------------------------------------------------
\subsection{$k_T$-factorization approach}
%--------------------------------------------------


At lowest order, the calculations with collinear photons  produce leptons that are back-to-back in transverse kinematics.
The transverse momentum appears at higher orders, however to describe full transverse momentum spectrum  one needs  
to match the calculations to resummation or dedicated parton shower algorithms. This approach is not considered in this paper.

In the $k_T$ factorization approach, one can parametrize the $\gamma^*p \rightarrow X$ vertices in terms of the proton structure functions. The photons from inelastic production have transverse momenta and non-zero virtualities $Q^2$ and the unintegrated photon distributions are used, in contrast to collinear distributions.
In the DIS limit, the unintegrated inelastic photon flux can be obtained using the following equation~\cite{daSilveira:2014jla, Luszczak:2015aoa}:

\begin{eqnarray}
\gamma^p_{inel}(x,Q^2) = {1\over x} 
{1 \over \pi Q^2} \, \int_{M^2_{\rm thr}} dM_X^2 {\cal{F}}^{\mathrm{in}}_{\gamma^* \leftarrow p} (x,\vec{q}_T^2,M^2_X) \, ,
\end{eqnarray}
%%
and we use the functions $ {\cal{F}}^{\mathrm{in}}_{\gamma^* \leftarrow p}$ from \cite{Budnev:1974de, Luszczak:2018ntp}:
%%%
\begin{eqnarray}
{\cal{F}}^{\mathrm{in}}_{\gamma^* \leftarrow p} (x,\vec{q}_T^2, M_X) &=& {\alpha_{\rm em} \over \pi} 
\Big\{(1-x) \Big( {\vec{q}_T^2 \over \vec{q}_T^2 + x (M_X^2 - m_p^2) + x^2 m_p^2  }\Big)^2  
{F_2(x_{\rm Bj},Q^2) \over Q^2 + M_X^2 - m_p^2}  \nonumber \\
&+& {x^2 \over 4 x^2_{\rm Bj}}  
{\vec{q}_T^2 \over \vec{q}_T^2 + x (M_X^2 - m_p^2) + x^2 m_p^2  }
{2 x_{\rm Bj} F_1(x_{\rm Bj},Q^2) \over Q^2 + M_X^2 - m_p^2} \Big\} \, .
\label{eq:flux_in}
\end{eqnarray}
%%%
The virtuality $Q^2$ of the photon depends on the photon transverse momentum ($\vec{q}_T^2$) and the proton remnant mass ($M_X$):
%%%
\begin{eqnarray}
Q^2 =  {\vec{q}_T^2 + x (M_X^2 - m_p^2) + x^2 m_p^2 \over (1-x)} \, .
\label{eq:q2}
\end{eqnarray}
%%
Moreover, the proton structure functions $F_1(x_{\rm Bj},Q^2)$ and $F_2(x_{\rm Bj},Q^2)$ require the argument
%%%
\begin{eqnarray}
x_{\rm Bj} = {Q^2 \over Q^2+M^2_X -m_p^2}.
\end{eqnarray}
%%
Note that in Eq.~\ref{eq:flux_in} instead of using $F_2(x_{\rm Bj},Q^2),F_1(x_{\rm Bj},Q^2)$, 
we in practice use the pair $F_2(x_{\rm Bj},Q^2),F_L(x_{\rm Bj},Q^2)$, where
%%%
\begin{eqnarray}
F_L(x_{\rm Bj},Q^2) = \Big( 1 + {4 x_{\rm Bj}^2 m_p^2 \over Q^2} \Big) F_2(x_{\rm Bj},Q^2) - 2 x_{\rm Bj} F_1(x_{\rm Bj},Q^2)
\end{eqnarray}
%%%
is the longitudinal structure function of the proton.

These unintegrated photon fluxes enter the $p+\textrm{Pb}\rightarrow \textrm{Pb} + \ell^+\ell^- + X$ production cross section as

\begin{equation}
\sigma = S^2 \int dx_p dx_{\rm Pb} d\vec{q}_T \Big[ \gamma^{p}_{el}(x_p, Q^2) + \gamma^{p}_{inel}(x_p,Q^2) \Big]
 \gamma^{\rm Pb}_{el}(x_{\rm Pb})
\sigma_{\gamma^{*}  \gamma \rightarrow \ell^+ \ell^-}(x_p, x_{\rm Pb}, \vec{q_T}) \,,
\label{kt_factorization_formula}
\end{equation}
%
where $\sigma_{\gamma^{*} \gamma \rightarrow \ell^+ \ell^-}$ is the off-shell elementary cross-section~\cite{daSilveira:2014jla} and  for  $x_p \ll 1$ we have $Q^2 \approx \vec{q}_T^2$ (see Eq.~\ref{eq:q2}).
In analogy to QCD $k_T$-factorization \cite{Catani:1990eg}, the same structures that enter
the $\gamma g^* \to q \bar q$ elementary cross-section calculation are present in $\sigma_{\gamma^{*} \gamma \rightarrow \ell^+ \ell^-}$.
One should note that while the fluxes do not depend on the direction of $\vec{q}_T$, averaging over directions
of $\vec{q}_T$ in the off-shell cross section replaces the average over photon polarizations in the collinear case.
