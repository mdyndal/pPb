\clearpage
%%%%%%%%%%%%%%%%%%%%%%%%%%%%%%%%%%
\section{Discussion}
%%%%%%%%%%%%%%%%%%%%%%%%%%%%%%%%%%

We take the opportunity to calculate expected number of events for realistic assumption on total integrated luminosity.
Based on the previous $p\textrm{Pb}$ runs at the LHC, we assume  $\int Ldt= 200~\textrm{nb}^{-1}$.
We also assume possible experimental efficiencies, manily due to trigger and recontruction of leptons, which we embed in a single correction factor $C=0.7$.

Table~\ref{fig:numbers} shows the expected number of events for $p+\textrm{Pb}\rightarrow \textrm{Pb} + \ell^+\ell^- + X$ production at $\sqrt{s_{N N}} = 8.16$~\TeV\ and configuration described above. 
Approximately 2500 elastic dilepton events are expected. 
Depending on the calculations, 3900 (collinear with LUXqed17 PDF) or 2400 ($k_T$-factorization with LUX-like $F_2+F_L$) reconstructed inelastic events are predicted. The difference between collinear and $k_T$-factorization can be therefore constrained with large significance, using existing datasets collected by ATLAS and CMS.

\begin{table}[t]
\begin{center}
\begin{tabular}{|l|c|c|}
\hline
Contribution & Expected events ($C=1$) & Expected events ($C=0.7$) \\
\hline
%$\gamma^{p}_{\rm{el}}$ [$b_{min}=0.7fm$] & 45.5(2) nb & 17.3(1) nb\\
%\hline
$\gamma^{p}_{\rm{el}}$  & 3600 & 2500\\ % [Electric]
\hline
$\gamma^{p}_{\rm{inel}}$ [LUXqed17 collinear] & 5600 & 3900 \\
\hline
$\gamma^{p}_{\rm{inel}}$ [LUX-like $F_2+F_L$] & 3400 & 2400 \\
\hline
\end{tabular}
\end{center}
\caption{Expected number of events for $p+\textrm{Pb}\rightarrow \textrm{Pb} + \ell^+\ell^- + X$ production at $\sqrt{s_{N N}} = 8.16$~\TeV\ assuming $\int Ldt= 200~\textrm{nb}^{-1}$. 
Shown are several contributions: purely elastic, inelastic with collinear LUXqed17 PDF and inelastic with $k_T$-factorization and LUX-like $F_2+F_L$ proton structure function parameterization.
An effect of possible experimental efficiencies is shown in last column.}
\label{fig:numbers}
\end{table}