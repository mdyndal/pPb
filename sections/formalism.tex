%%%%%%%%%%%%%%%%%%%%%%%%%%%%%%%%%%
\section{Formalism}
%%%%%%%%%%%%%%%%%%%%%%%%%%%%%%%%%%

%-------------------------------------
\subsection{Elastic photon fluxes}
%-------------------------------------


To get the distribution of the elastic photons from the proton, one can express the equivalent photon flux through
the electric and magnetic form factors $G_E(Q^2)$ and $G_M(Q^2)$ of the proton.
This contribution is obtained as
\begin{eqnarray}
   \gamma^{p}_{el}(x,Q^2) = \frac{\alpha_{\rm{em}}}{\pi}
\Big[ \Big( 1- {x \over 2} \Big)^2 \, {4 m_p^2 G_E^2(Q^2) + Q^2 G_M^2(Q^2) \over 4m_p^2 + Q^2} + {x^2 \over 4} G_M^2(Q^2) \Big]~,
\label{proton_el_flux}
\end{eqnarray}
where $x$ is the momentum fraction of the proton taken by the photon, $Q^2$ is the photon virtuality, $\alpha_{\rm{em}}$ is the electromagnetic structure constant and $m_p$ is the proton mass.

To express the elastic photon flux for the nucleus ($\gamma^{\rm Pb}_{el}$), we follow Ref.~\cite{Budnev:1974de} and replace 
\begin{eqnarray}
 {4 m_p^2 G_E^2(Q^2) + Q^2 G_M^2(Q^2) \over 4m_p^2 + Q^2} \longrightarrow Z^2 F_{\rm em}^2(Q^2)~,
 \end{eqnarray}
where $F_{\rm em}(Q^2)$ is the electromagnetic form factor of the nucleus and $Z$ is its charge.
We also neglect the magnetic form factor of the ion in the following.

For the Pb nucleus, we use the form factor parameterization from the STARlight MC generator~\cite{Klein:2016yzr}:
%%
\begin{eqnarray}
 F_{\rm em}(Q^2) = {3 \over (QR_A)^3}\Big[ \sin(QR_A) - QR_A \cos(QR_A) \Big] { 1 \over 1 + a^2 Q^2}~,
\end{eqnarray}
%%%
where $R_A = 1.1 A^{1/3}$ fm, $a = 0.7$ fm and $Q = \sqrt{Q^2}$.
%%%

The elastic photon PDFs of the proton and lead nucleus can be integrated over $Q^2$ as 
\begin{equation}
\gamma^{(p,Pb)}_{el}(x)  = \int d Q^2 \gamma^{(p,Pb)}_{el}(x, Q^2) \,.
\end{equation}
This is useful for the collinear-factorization approach since the $Q^2$ dependence factorizes in this case. 
%-----------------------------------------------------------
\subsection{Collinear-factorization approach and choice of the scale}
%-----------------------------------------------------------

The inelastic processes, with breakup of a proton, can be also considered.
At LO and at a given scale $\mu^2$, the photon parton distribution $\gamma^p_{inel}(x,\mu^2)$ of photons carrying a fraction $x$ of the proton's momentum, obeys the DGLAP equation:
%%
\begin{eqnarray}
{d \gamma^p_{inel}(x,\mu^2) \over d \log \mu^2} =&& {\alpha_{\rm{em}} \over 2 \pi} \int_x^1 {dy \over y} 
\Big [ \sum_q P_{\gamma \leftarrow q}(y) 
 q ({x \over y}, \mu^2 )   + P_{\gamma \leftarrow \gamma}(y) \gamma^p_{inel}({x \over y},\mu^2) \Big ]~,
\end{eqnarray}
%%
where $q (x,\mu^2)$ is the quark PDF,  $P_{\gamma \leftarrow q}$ is the $q\rightarrow\gamma$ splitting function, and $P_{\gamma \leftarrow \gamma}$ corresponds to the virtual self-energy correction to the photon propagator.
This is the basis for colinear photon-PDFs in the initial~\cite{Gluck:2002fi, Martin:2004dh} and more recent~\cite{Ball:2013hta, Martin:2014nqa, Schmidt:2014aba, Harland-Lang:2016kog, Giuli:2017oii, Manohar:2016nzj, Bertone:2017bme} analyses.

The computation of the photon-induced dilepton production cross section  requires definition of the  scale ($\mu^2$) at which the photon PDFs are convoluted.
The usual choice for $\mu$ is the mass of the system (motivated by the $s$-channel quark--antiquark annihilation process) or the transverse momentum of the leading object. 
These choices are however not optimal for the $t$- and $u$-channel initiated photon-induced process.
By analogy to DIS (Fig.~\ref{fig:diagrams}), where the scale is associated with the virtuality of the exchanged photon,
it is possible to define the scale in case of the $\gamma\gamma\rightarrow\ell^+\ell^-$ process.
This is achieved by taking the virtuality of the massive $t$- or $u$-channel propagator (Fig.~\ref{fig:diagrams}b or c).
Hence, $\mu^2 = -(p^{\gamma^{Pb}}-p^{\ell^-})^2$ for the $t$-channel diagram and $\mu^2 = -(p^{\gamma^{Pb}}-p^{\ell^+})^2$ for the $u$-channel exchange, where $p^{\gamma^{Pb}}$ is the four momentum
of the photon emitted by lead and $p^{\ell^{\pm}}$ is the four momentum of the lepton of the corresponding charge.
Note that the $u$- and $t$- channel diagrams have vanishing interference in the zero lepton mass limit. Therefore, they can be separated  while convoluting PDFs with the partonic cross section.

In the collinear approach, the $p+\textrm{Pb}\rightarrow \textrm{Pb} + \ell^+\ell^- + X$ production cross section can be written as
%
\begin{equation}
\sigma 
= S^2 \int dx_p dx_{\rm Pb} 
\Big [ \gamma^{p}_{el}(x_p) + \gamma^{p}_{inel}(x_p,\mu^2) \Big]
 \gamma^{\rm Pb}_{el}(x_{\rm Pb})
\sigma_{\gamma \gamma \rightarrow \ell^+ \ell^-}(x_p, x_{\rm Pb}) \,,
\label{collinear_factorization_formula}
\end{equation}
%
where $\sigma_{\gamma \gamma \rightarrow \ell^+ \ell^-}$
is the elementary cross section for the $\gamma \gamma \rightarrow \ell^+ \ell^-$ subprocess and $S^2$ is the so-called survival factor which takes into account the requirement that there be no hadronic interactions between the proton and the ion.

