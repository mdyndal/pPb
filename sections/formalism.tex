%%%%%%%%%%%%%%%%%%%%%%%%%%%%%%%%%%
\section{Formalism}
%%%%%%%%%%%%%%%%%%%%%%%%%%%%%%%%%%

%-------------------------------------
\subsection{Elastic vertices}
%-------------------------------------
In this work we are only interested in the elastic vertices on the nucleus side.

We recall, that for the proton, we can express the photon flux through
the electric and magnetic form factors $G_E(Q^2)$ and $G_M(Q^2)$ of the proton:
%%
\begin{eqnarray}
W^{\rm el}_T(M_X^2,Q^2) = \delta(M_X^2 - m_p^2) \, Q^2 G^2_M(Q^2) \, , \, W^{\rm el}_L(M_X^2, Q^2) = \delta(M_X^2 - m_p^2) 4m_p^2 G^2_E(Q^2) \, . 
\end{eqnarray}
%%
The contribution to the photon flux is then again obtained by contracting
%%
\begin{eqnarray}
  {p^\mu p^\nu \over s^2} \, W^{\rm el}_{\mu \nu}(M_X^2,Q^2) = \delta(M_X^2 - m_p^2) 
\Big[ \Big( 1- {z \over 2} \Big)^2 \, {4 m_p^2 G_E^2(Q^2) + Q^2 G_M^2(Q^2) \over 4m_p^2 + Q^2} + {z^2 \over 4} G_M^2(Q^2) \Big]
\nonumber \\
\end{eqnarray}
%%
For the nucleus, we follow \cite{Budnev:1974de}, and replace 
%%%
\begin{eqnarray}
 {4 m_p^2 G_E^2(Q^2) + Q^2 G_M^2(Q^2) \over 4m_p^2 + Q^2} \longrightarrow Z^2 F_{\rm em}^2(Q^2) \, .
 \end{eqnarray}
%%%
We neglect the magnetic form factor in the following. (It even rigorously vanishes for spinless nuclei.)

For the $^{208}\rm{Pb}$ nucleus, we use the realistic formfactor from the STARLIGHT MC.
%%
\begin{eqnarray}
 F_{\rm em}(Q^2) = {3 \over (QR_A)^3}\Big\{ \sin(QR_A) - QR_A \cos(QR_A) \Big\} { 1 \over 1 + a^2 Q^2} \, .
\end{eqnarray}
%%%
Here 
\begin{eqnarray}
R_A = 1.1 A^{1/3} \, {\rm fm} \, , \, a = 0.7 \, {\rm fm}, \, Q = \sqrt{Q^2} \,  .
\end{eqnarray}
%%%
Therefore we obtain the elastic flux
\begin{eqnarray}
{\cal{F}}^{{\rm{el}}}_{\gamma^* \leftarrow A} (z,\bq) = {Z^2 \alpha_{\rm{em}}\over \pi}  \, (1-z) \, 
\Big( {\bq^2 \over \bq^2 + z (M_X^2 - m_A^2) + z^2 m_A^2  }\Big)^2 \, F^2_{\rm em}(Q^2)
\, . \nonumber \\
\end{eqnarray}
%%
For $^{208}Pb$ the charge is $Z=82$.

%----------------------------------------
\subsection{Inelastic vertices}
%----------------------------------------

We consider the inelastic processes with breakup of a proton. 
Then the hadronic tensor is expressed in terms of the electromagnetic currents as:
%%
\begin{eqnarray}
 W^{\rm{in}}_{\mu \nu}(M_X^2,Q^2) = \overline{\sum_X} (2 \pi)^3 \, \delta^{(4)} (p_X - p_A - q) \, \bra{p} J_\mu \ket{X}\bra{X} J_\nu^\dagger \ket{p} \, d\Phi_X \, ,
\label{eq:Wmunu}
\end{eqnarray}
and its elements can be measured in inclusive electron scattering 
off the target. We wish to express it in terms of the virtual photoabsorption cross section
of transverse and longitudinal photons. To this end we introduce the covariant vectors tensors
%%
\begin{eqnarray}
e_\mu^{(0)} = \sqrt{Q^2 \over  X} \Big( p_{A\mu} - { (p_A \cdot q ) \over q^2} q_\mu \Big) \, , \, 
X = (p_A \cdot q)^2 + m_A^2 Q^2 \, , \, e^{(0)}\cdot e^{(0)} = + 1 \, ,
\end{eqnarray}
%%
and
%%
\begin{eqnarray}
\delta^\perp_{\mu \nu}(p_A,q) = g_{\mu \nu} - {q_\mu q_\nu \over q^2} - e^{(0)}_\mu e^{(0)}_\nu \, .
\end{eqnarray}
%%
Here $\delta^\perp_{\mu\nu}$ projects on photons carrying helicity $\pm 1$ in the $\gamma^* p$-cms frame,
and $e_\mu^{(0)}$ plays the role of the polarization vector of the longitudinal photon.
Notice that $q\cdot e^{0} = q^\mu \delta^\perp_{\mu \nu} = 0$, so that the hadronic tensor has the convenient
gauge invariant decomposition
%%
\begin{eqnarray}
  W^{\rm{in}}_{\mu \nu}(M_X^2,Q^2) = - \delta^\perp_{\mu \nu} (p_A,q) \, W^{\rm{in}}_T(M_X^2, Q^2) + e^{(0)}_\mu e^{(0)}_\nu \, W^{\rm{in}}_L(M_X^2, Q^2) \, .
\end{eqnarray}
The virtual photoabsorption cross sections are defined as
%%
\begin{eqnarray}
 \sigma_T(\gamma^* p) &=& {4 \pi \aem \over 4 \sqrt{X}} \, \Big(- {\delta^\perp_{\mu\nu} \over 2} \Big)  2\pi W^{\rm{in}}_{\mu \nu}(M_X^2,Q^2) \nonumber \\
 \sigma_L(\gamma^* p) &=& {4 \pi \aem \over 4 \sqrt{X}} \, e^{0}_\mu e^{0}_\nu \, 2 \pi W^{\rm{in}}_{\mu \nu}(M_X^2,Q^2) \, .
\end{eqnarray}
%%
It is customary to introduce dimensionless structure function $F_i(x_{\rm Bj},Q^2), i = T,L$ as
%%
\begin{eqnarray}
 \sigma_{T,L}(\gamma^* p) = {4 \pi^2 \aem \over Q^2} \, {1 \over \sqrt{1 + {4 x^2_{\rm Bj} m_A^2 \over Q^2}} } \, F_{T,L}(x_{\rm Bj},Q^2) \, ,
\end{eqnarray}
where
%%
\begin{eqnarray}
 x_{\rm Bj} = { Q^2 \over Q^2 + M_X^2 - m_A^2} \, .
\end{eqnarray}
%%
Then, our structure functions $W_{T,L}$ are expressed through the more conventional $F_{T,L}$ as
%%
\begin{eqnarray}
 W^{\rm{in}}_{T,L}(M_X^2,Q^2) = {1 \over x_{\rm Bj}} \, F_{T,L}(x_{\rm Bj},Q^2) \, . 
\end{eqnarray}
%%
In the literature one often finds rather $F_1(x_{\rm Bj}, Q^2), F_2(x_{\rm Bj},Q^2)$
structure functions, which are related to $F_{T,L}$ through
%%
\begin{eqnarray}
 F_T(x_{\rm Bj},Q^2) &=& 2x_{\rm Bj}  F_1(x_{\rm Bj},Q^2) \, , \nonumber \\
F_2(x_{\rm Bj},Q^2)  &=& { F_T(x_{\rm Bj},Q^2) +F_L(x_{\rm Bj},Q^2)
  \over 1 + {4 x^2_{\rm Bj} m_A^2 \over Q^2}} \; .
\end{eqnarray}
%%
Now, performing the contraction with $p^\mu_B p^\nu_B$, we get
%%
\begin{eqnarray}
  {p_B^\mu p_B^\nu \over s^2} \, W^{\rm in}_{\mu \nu}(M_X^2,Q^2) = \Big( 1 - {z \over x_{\rm Bj}} + {z^2\over 4 x_{\rm Bj}^2} \Big) {F_2(x_{\rm Bj},Q^2) \over Q^2 + M_X^2 - m_p^2} 
+ {z^2 \over 4 x^2_{\rm Bj}} { 2 x_{\rm Bj} F_1(x_{\rm Bj},Q^2) \over Q^2 + M_X^2 - m_p^2} \, .
\end{eqnarray}
%%
In the deep inelastic region $F_2 \sim F_T + F_L$, and using $2x_{\rm Bj} F_1 \sim F_2$ in the second term, we can write more succinctly
%%
\begin{eqnarray}
  {p_B^\mu p_B^\nu \over s^2} \, W^{\rm in}_{\mu \nu}(M_X^2,Q^2) = Q^2 \cdot f_T \Big( {z \over x_{\rm Bj}} \Big) \,  x_{\rm Bj} F_2(x_{\rm Bj},Q^2) \, ,
\end{eqnarray}
%%
with 
%%
\begin{eqnarray}
 f_T(y) = 1 - y + y^2/2 = {1 \over 2} \Big[ 1 + (1-y)^2 \Big] \, .
\end{eqnarray}

%-----------------------------------------------------------
\subsection{Collinear-factorization approach}
%-----------------------------------------------------------

Production of lepton pairs at large transverse momenta is a hard process, to which
standard arguments for factorization apply, and collinear factorization should be an appropriate starting point to calculate e.g. rapidity or transverse momentum spectra of leptons.
In fact, the dominant contribution to large-invariant mass dilepton pairs is of course the well known Drell-Yan process, but nothing prevents us from also including photon as partons along with quarks and gluons.

Then the photon parton distribution, $\gamma(z,Q^2)$, of photons carrying a fraction $z$ of the proton's
light-cone momentum, obeys the DGLAP equation,
%%
\begin{eqnarray}
{d \gamma(z,Q^2) \over d \log Q^2} =&& {\alpha_{\rm{em}} \over 2 \pi} \int_x^1 {dy \over y} 
\Big \{ \sum_f e_f^2 P_{\gamma \leftarrow q}(y) 
\Big[ q_f \Big({z \over y}, Q^2 \Big) + \bar q_f\Big({z \over y},Q^2\Big) \Big] \nonumber \\
&&+ P_{\gamma \leftarrow \gamma}(y) \gamma\Big({z \over y},Q^2\Big) \Big \} \, .
\end{eqnarray}
%%
In the complete set of DGLAP equations this photon density is then again coupled to the quark and antiquark
distributions:
%%
\begin{eqnarray}
 {d q_f(z,Q^2 )\over d \log Q^2} =&& {d q_f(z,Q^2) \over d \log Q^2}\Big|_{\rm{QCD}} + {\aem \over 2 \pi} \int_x^1 {dy \over y} 
\delta P_{q \leftarrow q}^{\rm{QED}}(y) q_f\Big({z \over y}, Q^2\Big)
\nonumber \\
+&& {\aem \over 2 \pi} \int_x^1 {dy \over y}  P_{q \leftarrow \gamma}(y)
\gamma\Big({z \over y},Q^2\Big) \; . \nonumber \\
\end{eqnarray}
%%
Due to the smallness of $\alpha_{\rm{em}}$ one would expect that the effect of photons on the quark and antiquark densities
can be safely neglected, unless one is interested in high order perturbative corrections to the QCD splitting functions
themselves.

Accordingly, we find two different approaches to DGLAP photons in the literature.

A first one, by Gl\"uck et al. \cite{Gluck:2002fi} asserts, that 
we can neglect the photon density on the right hand side of the evolution equations.
Then, at sufficiently large virtuality $Q_0^2$, the photon parton density can
be calculated from the collinear splitting of quarks and antiquarks 
$q \to q \gamma, \bar q \to \bar q  \gamma$. 
%%
\begin{eqnarray}
 {d \gamma(z,Q^2)\over d \log Q^2} = {\aem \over 2 \pi} \sum_f  e_f^2 \int_z^1 {dx \over x}   P_{\gamma \leftarrow q}\Big({z \over x}\Big)
\Big[ q_f(x,Q^2) + \bar q_f(x,Q^2) \Big] \; .
\label{eq:Dortmund}
\end{eqnarray}
%%
This equation is easily integrated, and gives the photon parton density as
%%
\begin{eqnarray}
 \gamma(z,Q^2) &=& \sum_f { \aem e_f^2 \over 2 \pi} \int_{Q_0^2}^{Q^2} {d \mu^2 \over \mu^2} 
\int_z^1 {dx \over x}  P_{\gamma \leftarrow q}\Big({z \over x}\Big)
\Big[ q_f(x,\mu^2) + \bar q_f(x,\mu^2) \Big] + \gamma(z,Q_0^2) \nonumber \\
&=& {\aem \over 2 \pi} \int_{Q_0^2}^{Q^2} {d \mu^2 \over \mu^2} \int_z^1 {dx \over x}  
P_{\gamma \leftarrow q}\Big({z \over x}\Big) {F_2(x,\mu^2) \over x} 
+ \gamma(z,Q_0^2) \, .
\end{eqnarray}
%%
One is left to specify -- from some  model considerations -- the photon density at some low scale $\gamma(z,Q_0^2)$, but
one may hope that at very large $Q^2 \gg Q_0^2 \sim 1 \, {\rm GeV}^2$ the part predicted perturbatively 
from quark and antiquark distributions dominates.

In addition to the above contribution from DGLAP splitting, Gl\"uck et
al. also add the Weizs\"acker-Williams flux from the coherent emission 
$p \to p \gamma^*$ without proton breakup as found in \cite{Budnev:1974de}.
 
The Durham \cite{Martin:2004dh,Martin:2014nqa} and NNPDF 
\cite{Ball:2013hta} groups have given a more involved treatment, 
in which the photon distribution is fully incorporated into the coupled 
DGLAP evolution equation.
As usual with DGLAP evolution, the photon parton density at a starting scale 
$\gamma(z,Q_0^2)$ needs to be specified. While \cite{Martin:2004dh,Martin:2014nqa} present model approaches, in Ref.\cite{Ball:2013hta} an ambitious attempt to
obtain $\gamma(z,Q_0^2)$ from a fit to experimental data is found.
Preliminary work by the CTEQ collaboration \cite{Schmidt:2014aba} is
also based on QED corrected DGLAP equations,
and attempts to fit the photon distribution from the prompt photon
production $e p \to \gamma e X$ at HERA
where in part of the phase space the Compton subprocess $e \gamma \to e \gamma$ contributes.

It should be noted, that in the approach of
\cite{Martin:2004dh,Martin:2014nqa}, the input distribution $\gamma(z,Q_0^2)$ contains the coherent --or elastic-- contribution with an intact proton in the final state.
Notice that due to the proton form factors the integral over
virtualities in the elastic case quickly converges, and the elastic
contribution is basically independent of $Q_0^2$, as soon as $Q_0^2 \gsim 0.7 \, \rm{GeV}^2$.


In the collinear approach the photon-photon contribution
to inclusive cross section for dilepton production can be written as:
%
\begin{equation}
{d \sigma^{(i,j)} \over d y_1 d y_2 d^2 p_T} 
= {1 \over 16 \pi^2 (x_1 x_2 s)^2}\sum_{i,j} 
x_1 \gamma^{(i)}(x_1,\mu^2) 
x_2 \gamma^{(j)}(x_2,\mu^2)
\overline{ |{\cal M}_{\gamma \gamma \rightarrow l^+ l^-}|^2 }.
\label{collinear_factorization_formula}
\end{equation}
%
Here 
%
\begin{eqnarray}
 x_1 &=&  \sqrt{p_T^2 + m_l^2 \over s} 
\Big( \exp(y_1) + \exp(y_2) \Big) \; , \nonumber \\
 x_2 &=&  \sqrt{p_T^2 + m_l^2 \over s} 
\Big( \exp(-y_1) + \exp(-y_2) \Big) \; . 
\end{eqnarray}
%%
Above indices $i$ and $j$ denote $i,j = \rm{el, in}$, i.e. they
correspond to elastic or inelastic components similarly as for 
the $k_T$-factorization discussed in section below.
The factorization scale is chosen as $\mu^2 = m_T^2 = p_T^2 + m_l^2$ (Renat?)

%--------------------------------------------------
\subsection{$k_T$-factorization approach}
%--------------------------------------------------

In this approach we start from the Feynman diagrams shown in Fig.\ref{fig:mateusz},
and exploit the high-energy kinematics.
Let the four-momenta of incoming protons be denoted $p_A,p_B$. At high energies 
the proton masses can be neglected, so that $p_A^2 = p_B^2 =0, \,  2 (p_A\cdot p_B) =s$.


The photon-fusion production mechanism in leptonic and hadronic reactions
is in great detail reviewed in \cite{Budnev:1974de}, where also many original
references can be found. In the most general form, the invariant cross section
is written as a convolution of density matrices of photons in the beam particles,
and helicity amplitudes for the $\gamma^* \gamma^* \to l^+ l^-$ process.
In a high energy limit, where dileptons carry only a small fraction of the 
total center-of-mass energy, the density-matrix structure can be very much
simplified, and there emerges a $k_T$-factorization representation of the
cross section \cite{daSilveira:2014jla}.

The unintegrated photon fluxes introduced in \cite{daSilveira:2014jla}
can be expressed in terms of the hadronic tensor as 
%%
\begin{eqnarray}
 {\cal{F}}^{{\rm{in.el}}}_{\gamma^* \leftarrow A} (z,\bq) = {\alpha_{\rm{em}}\over \pi}  \, (1-z) \, 
\Big( {\bq^2 \over \bq^2 + z (M_X^2 - m_A^2) + z^2 m_A^2  }\Big)^2  \, 
\cdot {p_B^\mu p_B^\nu \over s^2} \, W^{\rm{in,el}}_{\mu \nu}(M_X^2,Q^2) dM_X^2 \, . \nonumber \\
\end{eqnarray}
%%
These unintegrated fluxes enter the cross section for dilepton production as
%%
\begin{eqnarray}
 {d \sigma^{(i,j)} \over dy_1 dy_2 d^2\bp_1 d^2\bp_2} &&=  \int  {d^2 \bq_1 \over \pi \bq_1^2} {d^2 \bq_2 \over \pi \bq_2^2}  
 {\cal{F}}^{(i)}_{\gamma^*/A}(x_1,\bq_1) \, {\cal{F}}^{(j)}_{\gamma^*/B}(x_2,\bq_2) 
{d \sigma^*(p_1,p_2;\bq_1,\bq_2) \over dy_1 dy_2 d^2\bp_1 d^2\bp_2} \, , \nonumber \\ 
\label{eq:kt-fact}
\end{eqnarray}
%%
where the indices $i,j \in \{\rm{el}, \rm{in} \}$ denote elastic or inelastic final states.
The longitudinal momentum fractions of photons are obtained from the rapidities 
and transverse momenta of final state leptons as:
%%
\begin{eqnarray}
x_1 &=& \sqrt{ {\bp_1^2 + m_l^2 \over s}} e^{y_1} +  \sqrt{ {\bp_2^2 +
    m_l^2 \over s}} e^{y_2} 
\; , \nonumber \\
x_2 &=& \sqrt{ {\bp_1^2 + m_l^2 \over s}} e^{-y_1} +  \sqrt{ {\bp_2^2 + m_l^2 \over s}} e^{-y_2} \, .
\end{eqnarray}
%%
The explicit form of the off-shell cross section $d \sigma^*(p_1,p_2;\bq_1,\bq_2)/ dy_1 dy_2 d^2\bp_1 d^2\bp_2$ can be found in
Refs. \cite{daSilveira:2014jla,Luszczak:2015aoa}. 


%-------------------------------------------------------------------------------------------
\subsection{Structure functions as input for unintegrated fluxes}
%-------------------------------------------------------------------------------------------

The different parametrizations taken from the literature are labeled as:
%%
\begin{itemize} 

 \item ALLM \cite{Abramowicz:1991xz,Abramowicz:1997ms}: This
   parametrization gives a very good fit to $F_2$ in most of the measured region.
   
\item FJLLM \cite{Fiore:2002re}: This parametrization explicitly
  includes the nucleon resonances and gives an excellent fit of the CLAS data.
  
\item SY \cite{Suri:1971yx}: This paramerization of Suri and Yennie
    from the early 1970's does not include QCD-DGLAP evolution. It is
    still today often used as one of the defaults in the LPAIR event generator.
    
\item SU \cite{Szczurek:1999rd}: A parametrization which concentrates
   to give a good description at smallish and intermediate $Q^2$ at not too small $x$.
   A Vector-Meson-Dominance model inspired fit of $F_2$ at low $Q^2$, which is completed by the same NNLO MSTW structure function as above at large $Q^2$.
   
\item LUX-like: a newly constructed parametrization, which at $Q^2 > 9 \, \rm{GeV}^2$ uses an NNLO calculation of $F_2$ and $F_L$ from NNLO MSTW 2008 partons \cite{Martin:2009iq}. 
It employs a useful code by the MSTW group \cite{Martin:2009iq} to calculate structure functions. At $Q^2 > 9 \, \rm{GeV}^2$ this fit uses the parametrization of Bosted and Christy \cite{Bosted:2007xd} in the resonance region, and a version of the ALLM fit published by the HERMES Collaboration \cite{Airapetian:2011nu} for the continuum
region. It also uses information on the longitudinal structure function from SLAC \cite{Abe:1998ym}. As the fit is constructed closely following
the LUXqed work Ref.\cite{Manohar:2017eqh}.
\end{itemize} 
	
