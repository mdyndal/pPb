%%%%%%%%%%%%%%%%%%%%%%%%%%%%%%%%%%
\section{Formalism}
%%%%%%%%%%%%%%%%%%%%%%%%%%%%%%%%%%

%-------------------------------------
\subsection{Elastic photon fluxes}
%-------------------------------------
%In this work we are only interested in the elastic vertices on the nucleus side.

To get the distribution of the elastic photons from the proton, one can express the equivalent photon flux through
the electric and magnetic form factors $G_E(Q^2)$ and $G_M(Q^2)$ of the proton.
%%
%\begin{eqnarray}
%W^{\rm el}_T(M_X^2,Q^2) = \delta(M_X^2 - m_p^2) \, Q^2 G^2_M(Q^2) \, , \, W^{\rm el}_L(M_X^2, Q^2) = \delta(M_X^2 - m_p^2) 4m_p^2 G^2_E(Q^2) \, . 
%\end{eqnarray}
%%
This contribution is obtained as
%%
\begin{eqnarray}
  \gamma^{p}_{el}(x,Q^2) = \frac{\alpha_{\rm{em}}}{\pi}
\Big[ \Big( 1- {x \over 2} \Big)^2 \, {4 m_p^2 G_E^2(Q^2) + Q^2 G_M^2(Q^2) \over 4m_p^2 + Q^2} + {x^2 \over 4} G_M^2(Q^2) \Big]~,
\label{proton_el_flux}
\end{eqnarray}
%%
where $x$ is the momentum fraction of the proton taken by the photon, $Q^2$ is the photon virtuality, $\alpha_{\rm{em}}$ is the electromagnetic structure constant and $m_p$ is the proton mass.

To express the elastic photon flux for the nucleus ($\gamma^{\rm Pb}_{el}$), we follow Ref.~\cite{Budnev:1974de} and replace 
%%%
\begin{eqnarray}
 {4 m_p^2 G_E^2(Q^2) + Q^2 G_M^2(Q^2) \over 4m_p^2 + Q^2} \longrightarrow Z^2 F_{\rm em}^2(Q^2)~,
 \end{eqnarray}
%%%
where $F_{\rm em}(Q^2)$ is the electromagnetic formfactor of the nucleus and $Z$ is its charge.
We also neglect the magnetic formfactor of the ion in the following.
% (it even rigorously vanishes for spinless nuclei).

For the Pb nucleus, we use the formfactor parameterization from the STARlight MC generator~\cite{Klein:2016yzr}:
%%
\begin{eqnarray}
 F_{\rm em}(Q^2) = {3 \over (QR_A)^3}\Big[ \sin(QR_A) - QR_A \cos(QR_A) \Big] { 1 \over 1 + a^2 Q^2}~,
\end{eqnarray}
%%%
where $R_A = 1.1 A^{1/3}$ fm, $a = 0.7$ fm and $Q = \sqrt{Q^2}$.
%%%

%Therefore we obtain the elastic flux
%\begin{eqnarray}
%{\cal{F}}^{{\rm{el}}}_{\gamma^* \leftarrow A} (z,\bq) = {Z^2 \alpha_{\rm{em}}\over \pi}  \, (1-z) \, 
%\Big( {\bq^2 \over \bq^2 + z (M_X^2 - m_A^2) + z^2 m_A^2  }\Big)^2 \, F^2_{\rm em}(Q^2)
%\, . \nonumber \\
%\end{eqnarray}
%%%
%For $^{208}Pb$ the charge is $Z=82$.

%%----------------------------------------
%\subsection{Inelastic vertices}
%%----------------------------------------
%
%We also consider the inelastic processes with breakup of a proton. 
%Then the hadronic tensor is expressed in terms of the electromagnetic currents as:
%%%
%\begin{eqnarray}
% W^{\rm{in}}_{\mu \nu}(M_X^2,Q^2) = \overline{\sum_X} (2 \pi)^3 \, \delta^{(4)} (p_X - p_A - q) \, \bra{p} J_\mu \ket{X}\bra{X} J_\nu^\dagger \ket{p} \, d\Phi_X \, ,
%\label{eq:Wmunu}
%\end{eqnarray}
%and its elements can be measured in inclusive electron scattering 
%off the target. We wish to express it in terms of the virtual photoabsorption cross section
%of transverse and longitudinal photons. To this end we introduce the covariant vectors tensors
%%%
%\begin{eqnarray}
%e_\mu^{(0)} = \sqrt{Q^2 \over  X} \Big( p_{A\mu} - { (p_A \cdot q ) \over q^2} q_\mu \Big) \, , \, 
%X = (p_A \cdot q)^2 + m_A^2 Q^2 \, , \, e^{(0)}\cdot e^{(0)} = + 1 \, ,
%\end{eqnarray}
%%%
%and
%%%
%\begin{eqnarray}
%\delta^\perp_{\mu \nu}(p_A,q) = g_{\mu \nu} - {q_\mu q_\nu \over q^2} - e^{(0)}_\mu e^{(0)}_\nu \, .
%\end{eqnarray}
%%%
%Here $\delta^\perp_{\mu\nu}$ projects on photons carrying helicity $\pm 1$ in the $\gamma^* p$-cms frame,
%and $e_\mu^{(0)}$ plays the role of the polarization vector of the longitudinal photon.
%Notice that $q\cdot e^{0} = q^\mu \delta^\perp_{\mu \nu} = 0$, so that the hadronic tensor has the convenient
%gauge invariant decomposition
%%%
%\begin{eqnarray}
%  W^{\rm{in}}_{\mu \nu}(M_X^2,Q^2) = - \delta^\perp_{\mu \nu} (p_A,q) \, W^{\rm{in}}_T(M_X^2, Q^2) + e^{(0)}_\mu e^{(0)}_\nu \, W^{\rm{in}}_L(M_X^2, Q^2) \, .
%\end{eqnarray}
%The virtual photoabsorption cross sections are defined as
%%%
%\begin{eqnarray}
% \sigma_T(\gamma^* p) &=& {4 \pi \aem \over 4 \sqrt{X}} \, \Big(- {\delta^\perp_{\mu\nu} \over 2} \Big)  2\pi W^{\rm{in}}_{\mu \nu}(M_X^2,Q^2) \nonumber \\
% \sigma_L(\gamma^* p) &=& {4 \pi \aem \over 4 \sqrt{X}} \, e^{0}_\mu e^{0}_\nu \, 2 \pi W^{\rm{in}}_{\mu \nu}(M_X^2,Q^2) \, .
%\end{eqnarray}
%%%
%It is customary to introduce dimensionless structure function $F_i(x_{\rm Bj},Q^2), i = T,L$ as
%%%
%\begin{eqnarray}
% \sigma_{T,L}(\gamma^* p) = {4 \pi^2 \aem \over Q^2} \, {1 \over \sqrt{1 + {4 x^2_{\rm Bj} m_A^2 \over Q^2}} } \, F_{T,L}(x_{\rm Bj},Q^2) \, ,
%\end{eqnarray}
%where
%%%
%\begin{eqnarray}
% x_{\rm Bj} = { Q^2 \over Q^2 + M_X^2 - m_A^2} \, .
%\end{eqnarray}
%%%
%Then, our structure functions $W_{T,L}$ are expressed through the more conventional $F_{T,L}$ as
%%%
%\begin{eqnarray}
% W^{\rm{in}}_{T,L}(M_X^2,Q^2) = {1 \over x_{\rm Bj}} \, F_{T,L}(x_{\rm Bj},Q^2) \, . 
%\end{eqnarray}
%%%
%In the literature one often finds rather $F_1(x_{\rm Bj}, Q^2), F_2(x_{\rm Bj},Q^2)$
%structure functions, which are related to $F_{T,L}$ through
%%%
%\begin{eqnarray}
% F_T(x_{\rm Bj},Q^2) &=& 2x_{\rm Bj}  F_1(x_{\rm Bj},Q^2) \, , \nonumber \\
%F_2(x_{\rm Bj},Q^2)  &=& { F_T(x_{\rm Bj},Q^2) +F_L(x_{\rm Bj},Q^2)
%  \over 1 + {4 x^2_{\rm Bj} m_A^2 \over Q^2}} \; .
%\end{eqnarray}
%%%
%Now, performing the contraction with $p^\mu_B p^\nu_B$, we get
%%%
%\begin{eqnarray}
%  {p_B^\mu p_B^\nu \over s^2} \, W^{\rm in}_{\mu \nu}(M_X^2,Q^2) = \Big( 1 - {z \over x_{\rm Bj}} + {z^2\over 4 x_{\rm Bj}^2} \Big) {F_2(x_{\rm Bj},Q^2) \over Q^2 + M_X^2 - m_p^2} 
%+ {z^2 \over 4 x^2_{\rm Bj}} { 2 x_{\rm Bj} F_1(x_{\rm Bj},Q^2) \over Q^2 + M_X^2 - m_p^2} \, .
%\end{eqnarray}
%%%
%In the deep inelastic region $F_2 \sim F_T + F_L$, and using $2x_{\rm Bj} F_1 \sim F_2$ in the second term, we can write more succinctly
%%%
%\begin{eqnarray}
%  {p_B^\mu p_B^\nu \over s^2} \, W^{\rm in}_{\mu \nu}(M_X^2,Q^2) = Q^2 \cdot f_T \Big( {z \over x_{\rm Bj}} \Big) \,  x_{\rm Bj} F_2(x_{\rm Bj},Q^2) \, ,
%\end{eqnarray}
%%%
%with 
%%%
%\begin{eqnarray}
% f_T(y) = 1 - y + y^2/2 = {1 \over 2} \Big[ 1 + (1-y)^2 \Big] \, .
%\end{eqnarray}

%-----------------------------------------------------------
\subsection{Collinear-factorization approach and choice of the scale}
%-----------------------------------------------------------

%Production of lepton pairs at large transverse momenta is a hard process, to which
%standard arguments for factorization apply, and collinear factorization should be an appropriate starting point to calculate e.g. rapidity or transverse momentum spectra of leptons.
%In fact, the dominant contribution to large-invariant mass dilepton pairs is of course the well known Drell-Yan process, but nothing prevents us from also including photon as partons along with quarks and gluons.
The inelastic processes, with breakup of a proton, can be also considered.
At LO and at a given scale $\mu^2$, the photon parton distribution $\gamma^p_{inel}(x,\mu^2)$ of photons carrying a fraction $x$ of the proton's momentum, obeys the DGLAP equation:
%%
\begin{eqnarray}
{d \gamma^p_{inel}(x,\mu^2) \over d \log \mu^2} =&& {\alpha_{\rm{em}} \over 2 \pi} \int_x^1 {dy \over y} 
\Big [ \sum_q P_{\gamma \leftarrow q}(y) 
 q ({x \over y}, \mu^2 )   + P_{\gamma \leftarrow \gamma}(y) \gamma^p_{inel}({x \over y},\mu^2) \Big ]~,
\end{eqnarray}
%%
where $q (x,\mu^2)$ is the quark PDF, $e_q$ is the quark charge, $P_{\gamma \leftarrow q}$ is the $q\rightarrow\gamma$ splitting function, and $P_{\gamma \leftarrow \gamma}$ corresponds to the virtual self-energy correction to the photon propagator.
This is the basis for colinear photon-PDFs in the initial~\cite{Gluck:2002fi, Martin:2004dh} and more recent~\cite{Ball:2013hta, Martin:2014nqa, Schmidt:2014aba, Harland-Lang:2016kog, Giuli:2017oii, Manohar:2016nzj, Bertone:2017bme} analyses.

%In these studies the choice of the scale $\mu^2$, which is needed to calculate the cross section for the process of interest, is often not well-motivated. 
The computation of photon-induced dilepton production cross section  requires definition of the  scale ($\mu^2$) at which the photon PDFs are convoluted.
The usual choice for $\mu$ is the mass of the system (motivated by the $s$-channel quark--antiquark annihilation process) or the transverse momentum of the leading object. 
These choices are however not optimal for the $s$- and $t$-channel initiated photon-induced process.
By analogy to DIS (Fig.~\ref{fig:diagrams}), where the scale is associated with the virtuality of the exchanged photon,
it is possible to define the scale in case of the $\gamma\gamma\rightarrow\ell^+\ell^-$ process.
This is achieved by taking the virtuality of the massive $t$- or $u$-channel propagator (Fig.~\ref{fig:diagrams}b or c).
Hence, $\mu^2 = -(p^{\gamma^{Pb}}-p^{\ell^-})^2$ for the $t$-channel diagram and $\mu^2 = -(p^{\gamma^{Pb}}-p^{\ell^+})^2$ for the $u$-channel exchange, repsectively, where $p^{\gamma^{Pb}}$ is the four momentum
of the photon emitted by lead and $p^{\ell^{\pm}}$ is the four momentum of the lepton of the corresponding charge.
Note that the $u$- and $t$- channel diagrams have vanishing interference in the zero lepton mass limit. Therefore, they can be separated  while convoluting PDFs with the partonic cross section.
%In this case, on can use the average $p_T$ of the leptons with appropriate uncertainty.

%In the complete set of DGLAP equations this photon density is then again coupled to the quark and antiquark
%distributions:
%%%
%\begin{eqnarray}
% {d q_f(z,Q^2 )\over d \log Q^2} =&& {d q_f(z,Q^2) \over d \log Q^2}\Big|_{\rm{QCD}} + {\aem \over 2 \pi} \int_x^1 {dy \over y} 
%\delta P_{q \leftarrow q}^{\rm{QED}}(y) q_f\Big({z \over y}, Q^2\Big)
%\nonumber \\
%+&& {\aem \over 2 \pi} \int_x^1 {dy \over y}  P_{q \leftarrow \gamma}(y)
%\gamma\Big({z \over y},Q^2\Big) \; . \nonumber \\
%\end{eqnarray}
%%%
%Due to the smallness of $\alpha_{\rm{em}}$ one would expect that the effect of photons on the quark and antiquark densities
%can be safely neglected, unless one is interested in high order perturbative corrections to the QCD splitting functions
%themselves.
%
%Accordingly, we find two different approaches to DGLAP photons in the literature.
%
%A first one, by Gl\"uck et al. \cite{Gluck:2002fi} asserts, that 
%we can neglect the photon density on the right hand side of the evolution equations.
%Then, at sufficiently large virtuality $Q_0^2$, the photon parton density can
%be calculated from the collinear splitting of quarks and antiquarks 
%$q \to q \gamma, \bar q \to \bar q  \gamma$. 
%%%
%\begin{eqnarray}
% {d \gamma(z,Q^2)\over d \log Q^2} = {\aem \over 2 \pi} \sum_f  e_f^2 \int_z^1 {dx \over x}   P_{\gamma \leftarrow q}\Big({z \over x}\Big)
%\Big[ q_f(x,Q^2) + \bar q_f(x,Q^2) \Big] \; .
%\label{eq:Dortmund}
%\end{eqnarray}
%%%
%This equation is easily integrated, and gives the photon parton density as
%%%
%\begin{eqnarray}
% \gamma(z,Q^2) &=& \sum_f { \aem e_f^2 \over 2 \pi} \int_{Q_0^2}^{Q^2} {d \mu^2 \over \mu^2} 
%\int_z^1 {dx \over x}  P_{\gamma \leftarrow q}\Big({z \over x}\Big)
%\Big[ q_f(x,\mu^2) + \bar q_f(x,\mu^2) \Big] + \gamma(z,Q_0^2) \nonumber \\
%&=& {\aem \over 2 \pi} \int_{Q_0^2}^{Q^2} {d \mu^2 \over \mu^2} \int_z^1 {dx \over x}  
%P_{\gamma \leftarrow q}\Big({z \over x}\Big) {F_2(x,\mu^2) \over x} 
%+ \gamma(z,Q_0^2) \, .
%\end{eqnarray}
%%%
%One is left to specify -- from some  model considerations -- the photon density at some low scale $\gamma(z,Q_0^2)$, but
%one may hope that at very large $Q^2 \gg Q_0^2 \sim 1 \, {\rm GeV}^2$ the part predicted perturbatively 
%from quark and antiquark distributions dominates.
%
%In addition to the above contribution from DGLAP splitting, Gl\"uck et
%al. also add the Weizs\"acker-Williams flux from the coherent emission 
%$p \to p \gamma^*$ without proton breakup as found in \cite{Budnev:1974de}.
% 
%The Durham \cite{Martin:2004dh,Martin:2014nqa} and NNPDF 
%\cite{Ball:2013hta} groups have given a more involved treatment, 
%in which the photon distribution is fully incorporated into the coupled 
%DGLAP evolution equation.
%As usual with DGLAP evolution, the photon parton density at a starting scale 
%$\gamma(z,Q_0^2)$ needs to be specified. While \cite{Martin:2004dh,Martin:2014nqa} present model approaches, in Ref.\cite{Ball:2013hta} an ambitious attempt to
%obtain $\gamma(z,Q_0^2)$ from a fit to experimental data is found.
%Preliminary work by the CTEQ collaboration \cite{Schmidt:2014aba} is
%also based on QED corrected DGLAP equations,
%and attempts to fit the photon distribution from the prompt photon
%production $e p \to \gamma e X$ at HERA
%where in part of the phase space the Compton subprocess $e \gamma \to e \gamma$ contributes.
%
%It should be noted, that in the approach of
%\cite{Martin:2004dh,Martin:2014nqa}, the input distribution $\gamma(z,Q_0^2)$ contains the coherent --or elastic-- contribution with an intact proton in the final state.
%Notice that due to the proton form factors the integral over
%virtualities in the elastic case quickly converges, and the elastic
%contribution is basically independent of $Q_0^2$, as soon as $Q_0^2 \gsim 0.7 \, \rm{GeV}^2$.


In the collinear approach, the $p+\textrm{Pb}\rightarrow \textrm{Pb} + \ell^+\ell^- + X$ production cross section can be written as
%
\begin{equation}
\sigma 
= S^2 \int dx_p dx_{\rm Pb} \Big[
\left ( \gamma^{p}_{el}(x_p) + \gamma^{p}_{inel}(x_p,\mu^2) \right)
 \gamma^{\rm Pb}_{el}(x_{\rm Pb})
\sigma_{\gamma \gamma \rightarrow \ell^+ \ell^-}(x_p, x_{\rm Pb})  \Big]~,
\label{collinear_factorization_formula}
\end{equation}
%
where $\sigma_{\gamma \gamma \rightarrow \ell^+ \ell^-}$
is the elementary cross section for the $\gamma \gamma \rightarrow \ell^+ \ell^-$ subprocess and $S^2$ is the so-called survival factor which takes into account the requirement that there be no hadronic interactions between the proton and the ion.

%Here 
%%
%\begin{eqnarray}
% x_1 &=&  \sqrt{p_T^2 + m_l^2 \over s} 
%\Big( \exp(y_1) + \exp(y_2) \Big) \; , \nonumber \\
% x_2 &=&  \sqrt{p_T^2 + m_l^2 \over s} 
%\Big( \exp(-y_1) + \exp(-y_2) \Big) \; . 
%\end{eqnarray}
%%%
%Above indices $i$ and $j$ denote $i,j = \rm{el, in}$, i.e. they
%correspond to elastic or inelastic components similarly as for 
%the $k_T$-factorization discussed in section below.
%The factorization scale is chosen as $\mu^2 = m_T^2 = p_T^2 + m_l^2$ (Renat?)
