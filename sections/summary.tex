%%%%%%%%%%%%%%%%%%%%%%%%%%%%%%%%%%
\section{Summary}
%%%%%%%%%%%%%%%%%%%%%%%%%%%%%%%%%%

In summary, we propose a method that would provide an unambiguous test of the photon parton distribution at LHC energies, and allow 
constraints to be placed on it.
This method is based on the measurement of the cross-section for the reaction $p+\textrm{Pb}\rightarrow \textrm{Pb} + \ell^+\ell^- + X$, where the expected background is small compared to the analogous process in $pp$ collisions. 
Results are shown for different choices of collinear photon PDFs, and a comparison is made with unintegrated photon distributions that include non-zero photon transverse momentum.
Due to the smearing of dilepton transverse momentum introduced by the $k_T$-factorization approach, these two approaches lead to the cross sections that differ by about 30\%.
Moreover, for collinear approach and  by analogy to DIS, an optimal choice of the scale is identified.
Using simple (realistic) experimental requirements on lepton kinematics, it is shown that one can expect O(3000) inelastic events with the existing datasets recorded by ATLAS/CMS at $\sqrt{s_{N N}} = 8.16$~\TeV\ for each lepton flavour.


%%%%%%%%%%%%%%%%%%%%%%%%%%%%%%%%%%
\section*{Acknowledgements}
We would like to thank James Ferrando for useful suggestions.
The work of M.L. was partially supported by the Center for Innovation and
Transfer of Natural Sciences and Engineering Knowledge in Rzesz{\'o}w.
The work of R.S. was partially supported by the BMBF-JINR cooperation.
M.L. and R.S. acknowledge the hospitality of DESY where a portion of
this work was performed.
%%%%%%%%%%%%%%%%%%%%%%%%%%%%%%%%%%
